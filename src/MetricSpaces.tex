\section{Metric Spaces: Examples and Definitions}

A \emph{metric space} is a set equipped with a notion of distance between points. Such
spaces arise in many contexts. We begin by considering a few examples.

\vspace{0.2in}
\noindent Examples:

\begin{enumerate}
\item
\item
\end{enumerate}

\begin{defn}
Let X be a non-empty set and $x, y \in X$ . Let $d: X\times X \leftarrow \mathbb{R}$
such that it satisfies
\begin{enumerate}
    \item Positivity:
    \begin{enumerate}
        \item $d(x,y) \geq 0$
        \item $d(x,x) = 0$
        \item $d(x,y) = 0 \ \Rightarrow\ x = y$
    \end{enumerate}
    \item Symmetry:  $d(x,y) = d(y, x)$
    \item Triangle Inequality: $d(x,y)+d(y,z) \geq d(x,z)$
\end{enumerate}
Then $X$ is called a metric space with the metric $d$.
\begin{rmk}
    We would often like to allow $d(x,y)=\infty$ so $d:X\times X \Rightarrow [0, \infty]$
\end{rmk}

\begin{exr}
Let $d_{n} : X\times X \rightarrow \mathbb{R}$ is a metric on X, $\forall n \in \mathbb{N}$.
Prove that $d(x, y) = {inf}\limits_{n \in \mathbb{N}}\{d_{n}(x,y)\}$ satisfies
all conditions except 1(c).
\end{exr}

\begin{exr}
Let $d : X\times X \rightarrow \mathbb{R}$ satisfies all properties of a metric
expect 1(c).
    \begin{enumerate}
        \item Show that the relation
            $$x \sim y \Leftrightarrow d(x,y) = 0$$
          is an equivalence realtion on X.

        \item Show that there is a well defined function
          $\bar{d} : \bigslant{X}{\sim} \times \bigslant{X}{\sim} \rightarrow
          \mathbb{R} $ given by
            $$ \bar{d}([x], [y]) = d(x, y) $$
          where $[x]$ is an equivalence class of $X$ under $\sim$.

        \item Show that $\bar{d}$ is a metric
    \end{enumerate}

\end{exr}

\end{defn}
