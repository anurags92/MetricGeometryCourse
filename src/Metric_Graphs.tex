\documentclass[a4paper,10pt]{book}


\usepackage{textcomp}

\usepackage{amsmath}

\usepackage{amsthm}

\usepackage[top=0.5in, bottom=0.5in, left=0.5in, right=0.5in]{geometry}

\usepackage{setspace}



\newtheorem{theorem}{Theorem}[chapter]

\newtheorem{lemma}[theorem]{Lemma}



\theoremstyle{definition}

\newtheorem{definition}[theorem]{Definition}

\newtheorem{example}[theorem]{Example}

\newtheorem{xca}[theorem]{Exercise}

\newtheorem{con}[theorem]{}



\theoremstyle{remark}

\newtheorem{remark}[theorem]{Remark}



\numberwithin{section}{chapter}

\numberwithin{equation}{chapter}



\begin{document}
 
\chapter{Metric Graphs}
 

\section{Graphs: Combinatorical Structure}
 
 
\begin{definition}
  
A graph \emph{$\Gamma$} consists of
  
  
\begin{itemize}
   
\item A pair of sets $(V,\ E)$ (called the `Vertex' and `Edge' sets respectively)
   
\item An involution  $\:\phi \colon E \rightarrow E$ satisfying $\phi (e) = \bar{e} $ and $\phi\circ\phi = \mathbf{1}_E$ (identity function)
  \item Functions $i,\ \tau: E \rightarrow V$ such that $i(\bar{e}) = \tau (e) $ and $\tau(\bar{e}) = i(e)$
  \end{itemize}

 \end{definition}

 Insert EXAMPLE 'ERE
 
 \section{A Graph as a Space}

   The following construction associates to a graph $\Gamma$, a space $| \Gamma |$ 
   \vspace{0.1in} \\ Let $ \xi = E \times [0,1] = \{ (e,t) : e \in E,\ t \in [0,1]\} $.\\
   Let $ \widetilde{\Gamma} =  \xi \cup V $ \\
   Let `\texttildelow' be the equivalence relation on $ \widetilde{\Gamma} $ generated by: 
   \begin{itemize}
      \item If $ (e,t) \in \xi $, then $ (e,t) \sim (\bar{e},1-t) $
      \item $ \forall \ e \in E $, $ i(e) \sim (e,0) $ and $ \tau (e) \sim (e,1) $
   \end{itemize}
   $ |\Gamma| $ is the quotient space $ \Gamma / \sim $
   \section{ Metric structure on a Graph}
   
   \subsection{Distance on the Vertex set V}
   \begin{definition}
      Two vertices $ v_{1}$ and $v_{2} $ are said to be \emph{adjacent} if there exists $ e \in E $ such that $ i(e) = v_1 $ and $ \tau (e) = v_2 $.
   \end{definition}
   Let $ \cal{D} = \{ $ d $: V \times V \rightarrow \mathbf{R} \ : \ $ 
 d is a metric; d($ v_1,\ v_2 $) $ \leq $ 1 if $ v_1 $, $ v_2 $ are adjacent $ \} $
   \begin{lemma}
      $ \cal{D} $ is not empty
   \end{lemma}
   \begin{definition}
      The \emph{Graph Metric} on V is defined as
      \begin{equation}
	  d_{max}(v_1,v_2)= \sup_{d \in \cal{D} }d(v_1,v_2) \nonumber
      \end{equation} 
   \end{definition}
   \begin{lemma}
	$ d_{max} $ is a metric.
    \proof 
    \begin{enumerate}
      \item \[\forall \ x \in V,\ d_{max}(x, x)= \sup_{d \in \cal{D}}d(x,x) = \sup\{ 0 \} = 0 \]
      \item \[\forall \ x,\ y \in V,\ d_{max}(x, y)= \sup_{d \in \cal{D} }d(x,y)= \sup_{d \in \cal{D} }d(y,x)  =d _{max}(y, x) \]
      \item \[ d_{max}(x, z)=\sup\{d(x,y): d \in \mathcal{D}\} \leq \sup_{}\{ d(x,y): d \in \mathcal{D}\} {( d \in \cal{D} \text{ is a metric})} \]
    \end{enumerate}    
  \end{lemma}
  \begin{definition}
   An \emph{edge path} is a ordered list (may be null, in whuch case it is a single vertex) of edges- denoted by $ \eta=(e_1,e_2,\dots,e_n) $, such that, $ \forall i,\ 1\leq i < n,\ i(e_i)=\tau(e_{i+1})$. $\eta$ is called ``an edge path from $i(e_1)$ to $\tau(e_n)$''.
  \end{definition}
  \begin{definition}
   An edge path is a \emph{loop} if $i(e_1) = \tau(e_n)$ 
  \end{definition}
  \begin{definition}
   An edge path is said to be \emph{reduced} if $\forall\ i,\ 1\leq i<n,\ e_i \neq \bar{e}_{i+1}$
  \end{definition}
  \begin{definition}
   A graph $\Gamma$ is \emph{connected} if $\forall\ x,y \in V,\ \exists$ an edge path from x to y.
  \end{definition}
  \begin{definition}
   A connected graph is called a \emph{tree} if it has no reduced edge loops.
  \end{definition}
  \begin{theorem}
   Let $\Gamma$ be a connected graph. Then for x, y $\in V$,
   \begin{equation}
    d_{max}(x,y)=min\{n\geq 0 :\ \exists\ \text{an edge path from x to y of cardinality }n\}
   \end{equation}
  \proof
  Let $d_{\Gamma}(x,y)=min\{n\geq 0 :\ \exists\ \text{an edge path from x to y of cardinality }n\}$.\\
  We shall show:
  \begin{enumerate}
  
   \item $d_\Gamma$ is a metric
   \item $d_\Gamma (i(e),\tau (e))\leq 1$ 
   \item $d_{max}(x,y) \leq d_\Gamma(x,y)$
  \end{enumerate}
Now, the first and second points establish that $d_\Gamma \in \mathcal{D} $. Then, $d_{max}$ being the supremum value will imply $d_{max}(x,y) \geq d_\Gamma (x,y)$. Together with 3, this will give the desired result. 
  \begin{lemma}
   $d_\Gamma$ is a metric
   \proof
   \begin{enumerate}
    \item \[d_\Gamma(x,x) = 0\] because the single vertex `x' is a null list of edges that give the minimal connecting edge path.
    \item \[d_\Gamma(x,y) = d_\Gamma(y,x)\] Now, in an (undirected) graph, $\eta$ is a valid edge path iff $\bar{\eta}$ is also a valid edge path.
    \begin{remark}
     \[ \eta = (e_1,e_2,\dots,e_n) \Rightarrow \bar{\eta}=(\bar{e}_n,\bar{e}_{n-1},\dots,\bar{e}_1)\] 
     If $\eta$ is an edge path from x to y, $\bar{\eta}$ is an edge path fom y to x.
    \end{remark}
    Hence, the involution of the minimal edge path connecting x to y will connect y to x. This will also be the minimal edge path connecting y to x since otherwise the involution of the shorter path (from y to x) will connect x to y and be shorter hat the oiginal path, which leads to a contradiction.
    \item If \[
              \eta_1=(e_1,\dots,e_n) \text{ is the minimal path connecting x to y}\] and\[
              \eta_2=(e_{n+1},\dots,e_m) \text{ is the minimal path connecting y to z}
             \]
             \begin{remark}
              Edges in $\eta_1$ and $\eta_2$ need not be distinct.
             \end{remark}
             Then,\[
                   \eta_1 \circ \eta_2 = (e_1,e_2,\dots,e_n,e_{n+1},\dots,e_m) \text{ is an edge path connecting x to z}
                  \]
                  Hence the minimal edge path connecting x to z will have to be, by definition, shorter than or the same length as this path. Which means
                  \[
                   d_{max}(x,z)=|\eta_{min}|\leq |\eta_1 \circ \eta_2| =m+n =d_{max}(x,y)+d_{max}(y,z)
                  \]
                     \end{enumerate}
    This proves that $d_{max}$ is a metric.
  \end{lemma}
  \begin{lemma}
    $d_\Gamma (i(e),\tau (e))\leq 1$
    \proof Since $(e)$ is an edge path connecting $i(e)$ and $\tau(e)$, the minimal edge path will have cardinality less than or equal to this. That is,
    \[
      d_{max}(i(e),\tau(e))\leq |(e)| = 1
    \]
  \end{lemma}
    \begin{lemma}
     $d_{max}(x,y) \leq d_\Gamma(x,y)$
    \proof Let $ d_\Gamma(x,y) = n$.
    This means that there exists an edge path $\eta = (e_1,\dots,e_n) $ connecting x to y. Now,
    \[
     d_{max}(x,y) \leq d_{max}(i(e_1,\tau(e_1)) + \dots + d_{max}(i(e_n),\tau(e_n)) \ \ \ \ \ \ \ \ \ \  \text{[$d_{max}$ is a metric]}
    \]
    \[
     \leq 1 + \dots +1=n \hspace{2in} [d_{max} \in \mathcal{D}]
    \]


  \end{lemma}

  \end{theorem}
\end{document}
